% ============================================================
% ONTO Paper - Sanitized Claims & Limitations
% Include with: % ============================================================
% ONTO Paper - Sanitized Claims & Limitations
% Include with: % ============================================================
% ONTO Paper - Sanitized Claims & Limitations
% Include with: % ============================================================
% ONTO Paper - Sanitized Claims & Limitations
% Include with: \input{claims_limitations}
% ============================================================

% ============================================================
% SANITIZED CLAIMS (use these exact phrasings)
% ============================================================

% MAIN CLAIM (Abstract/Intro)
\newcommand{\mainclaim}{%
ONTO significantly improves epistemic calibration on curated benchmark tasks, 
achieving substantially higher unknown detection recall compared to baseline LLMs.%
}

% CONTRIBUTION CLAIMS (Introduction)
\newcommand{\contribA}{%
We present ONTO, an ontological reasoning framework that augments LLM inference 
with explicit epistemic structure for improved uncertainty identification.%
}

\newcommand{\contribB}{%
We introduce ONTO-Bench, a benchmark for evaluating epistemic calibration 
comprising samples with explicit epistemic labels sourced from authoritative references.%
}

\newcommand{\contribC}{%
We demonstrate that explicit epistemic structure enables substantially higher 
unknown detection recall on our benchmark compared to baseline LLMs.%
}

% RESULTS CLAIM (avoid "dramatically", "revolutionary", etc.)
\newcommand{\resultsclaim}{%
ONTO achieves 96\% recall on unknown detection compared to less than 10\% 
for baseline LLMs on ONTO-Bench, suggesting that explicit epistemic structure 
can meaningfully improve uncertainty identification.%
}

% HEDGED CLAIMS (for discussion)
\newcommand{\hedgedclaim}{%
Our results suggest that ontological structure may provide a useful signal 
for epistemic calibration, though further work is needed to evaluate 
generalization to broader domains and real-world deployment scenarios.%
}

% ============================================================
% LIMITATIONS SECTION
% ============================================================

\section{Limitations}
\label{sec:limitations}

We acknowledge several limitations of our work:

\textbf{Dataset Construction.}
ONTO-Bench relies partially on curated questions from Wikipedia lists and 
authoritative sources. While we validate labels through cross-checking, 
the dataset may not capture the full distribution of epistemic challenges 
encountered in real-world applications. The ``unknown'' category is 
operationalized through curated open problems rather than a complete 
epistemological theory.

\textbf{Heuristic Detection.}
The ONTO epistemic oracle uses pattern-based heuristics for unknown detection 
rather than learned representations. While effective on our benchmark, this 
approach may not generalize to novel question formulations. We frame our 
contribution as demonstrating the value of explicit epistemic structure 
rather than proposing an optimal detection method.

\textbf{Baseline Configuration.}
We evaluate baseline LLMs without external tool augmentation (e.g., web search, 
retrieval) to ensure comparable settings with ONTO's core reasoning. 
Performance may differ with tool-augmented configurations.

\textbf{Dataset Scale.}
With 268 samples, ONTO-Bench is smaller than some established benchmarks. 
We prioritize label quality over quantity, but larger-scale evaluation 
would strengthen generalization claims.

\textbf{Evaluation Scope.}
We evaluate epistemic calibration through proxy classification tasks. 
Downstream impact on user trust, decision-making, and harm reduction 
requires separate investigation.

% ============================================================
% ETHICS STATEMENT
% ============================================================

\section{Ethics Statement}

\textbf{Intended Use.}
ONTO-Bench is designed for research evaluation of AI epistemic capabilities. 
It should not be used as the sole criterion for deployment decisions.

\textbf{Potential Misuse.}
High performance on ONTO-Bench does not guarantee safe deployment. 
Systems should undergo comprehensive evaluation before real-world use.

\textbf{Dataset Bias.}
Our dataset reflects Western scientific consensus and English-language sources. 
Epistemic norms may vary across cultures and languages.

% ============================================================
% REPRODUCIBILITY STATEMENT
% ============================================================

\section{Reproducibility Statement}

We provide complete materials for reproduction:

\begin{itemize}
    \item \textbf{Dataset}: Version-locked with SHA256 hash (see Appendix)
    \item \textbf{Code}: Full evaluation pipeline released
    \item \textbf{Baselines}: Model versions and prompts documented
    \item \textbf{Metrics}: Definitions and computation code provided
    \item \textbf{Random Seed}: Fixed at 42 for all experiments
\end{itemize}

Verification command: \texttt{python scripts/dataset\_version.py --verify}

% ============================================================

% ============================================================
% SANITIZED CLAIMS (use these exact phrasings)
% ============================================================

% MAIN CLAIM (Abstract/Intro)
\newcommand{\mainclaim}{%
ONTO significantly improves epistemic calibration on curated benchmark tasks, 
achieving substantially higher unknown detection recall compared to baseline LLMs.%
}

% CONTRIBUTION CLAIMS (Introduction)
\newcommand{\contribA}{%
We present ONTO, an ontological reasoning framework that augments LLM inference 
with explicit epistemic structure for improved uncertainty identification.%
}

\newcommand{\contribB}{%
We introduce ONTO-Bench, a benchmark for evaluating epistemic calibration 
comprising samples with explicit epistemic labels sourced from authoritative references.%
}

\newcommand{\contribC}{%
We demonstrate that explicit epistemic structure enables substantially higher 
unknown detection recall on our benchmark compared to baseline LLMs.%
}

% RESULTS CLAIM (avoid "dramatically", "revolutionary", etc.)
\newcommand{\resultsclaim}{%
ONTO achieves 96\% recall on unknown detection compared to less than 10\% 
for baseline LLMs on ONTO-Bench, suggesting that explicit epistemic structure 
can meaningfully improve uncertainty identification.%
}

% HEDGED CLAIMS (for discussion)
\newcommand{\hedgedclaim}{%
Our results suggest that ontological structure may provide a useful signal 
for epistemic calibration, though further work is needed to evaluate 
generalization to broader domains and real-world deployment scenarios.%
}

% ============================================================
% LIMITATIONS SECTION
% ============================================================

\section{Limitations}
\label{sec:limitations}

We acknowledge several limitations of our work:

\textbf{Dataset Construction.}
ONTO-Bench relies partially on curated questions from Wikipedia lists and 
authoritative sources. While we validate labels through cross-checking, 
the dataset may not capture the full distribution of epistemic challenges 
encountered in real-world applications. The ``unknown'' category is 
operationalized through curated open problems rather than a complete 
epistemological theory.

\textbf{Heuristic Detection.}
The ONTO epistemic oracle uses pattern-based heuristics for unknown detection 
rather than learned representations. While effective on our benchmark, this 
approach may not generalize to novel question formulations. We frame our 
contribution as demonstrating the value of explicit epistemic structure 
rather than proposing an optimal detection method.

\textbf{Baseline Configuration.}
We evaluate baseline LLMs without external tool augmentation (e.g., web search, 
retrieval) to ensure comparable settings with ONTO's core reasoning. 
Performance may differ with tool-augmented configurations.

\textbf{Dataset Scale.}
With 268 samples, ONTO-Bench is smaller than some established benchmarks. 
We prioritize label quality over quantity, but larger-scale evaluation 
would strengthen generalization claims.

\textbf{Evaluation Scope.}
We evaluate epistemic calibration through proxy classification tasks. 
Downstream impact on user trust, decision-making, and harm reduction 
requires separate investigation.

% ============================================================
% ETHICS STATEMENT
% ============================================================

\section{Ethics Statement}

\textbf{Intended Use.}
ONTO-Bench is designed for research evaluation of AI epistemic capabilities. 
It should not be used as the sole criterion for deployment decisions.

\textbf{Potential Misuse.}
High performance on ONTO-Bench does not guarantee safe deployment. 
Systems should undergo comprehensive evaluation before real-world use.

\textbf{Dataset Bias.}
Our dataset reflects Western scientific consensus and English-language sources. 
Epistemic norms may vary across cultures and languages.

% ============================================================
% REPRODUCIBILITY STATEMENT
% ============================================================

\section{Reproducibility Statement}

We provide complete materials for reproduction:

\begin{itemize}
    \item \textbf{Dataset}: Version-locked with SHA256 hash (see Appendix)
    \item \textbf{Code}: Full evaluation pipeline released
    \item \textbf{Baselines}: Model versions and prompts documented
    \item \textbf{Metrics}: Definitions and computation code provided
    \item \textbf{Random Seed}: Fixed at 42 for all experiments
\end{itemize}

Verification command: \texttt{python scripts/dataset\_version.py --verify}

% ============================================================

% ============================================================
% SANITIZED CLAIMS (use these exact phrasings)
% ============================================================

% MAIN CLAIM (Abstract/Intro)
\newcommand{\mainclaim}{%
ONTO significantly improves epistemic calibration on curated benchmark tasks, 
achieving substantially higher unknown detection recall compared to baseline LLMs.%
}

% CONTRIBUTION CLAIMS (Introduction)
\newcommand{\contribA}{%
We present ONTO, an ontological reasoning framework that augments LLM inference 
with explicit epistemic structure for improved uncertainty identification.%
}

\newcommand{\contribB}{%
We introduce ONTO-Bench, a benchmark for evaluating epistemic calibration 
comprising samples with explicit epistemic labels sourced from authoritative references.%
}

\newcommand{\contribC}{%
We demonstrate that explicit epistemic structure enables substantially higher 
unknown detection recall on our benchmark compared to baseline LLMs.%
}

% RESULTS CLAIM (avoid "dramatically", "revolutionary", etc.)
\newcommand{\resultsclaim}{%
ONTO achieves 96\% recall on unknown detection compared to less than 10\% 
for baseline LLMs on ONTO-Bench, suggesting that explicit epistemic structure 
can meaningfully improve uncertainty identification.%
}

% HEDGED CLAIMS (for discussion)
\newcommand{\hedgedclaim}{%
Our results suggest that ontological structure may provide a useful signal 
for epistemic calibration, though further work is needed to evaluate 
generalization to broader domains and real-world deployment scenarios.%
}

% ============================================================
% LIMITATIONS SECTION
% ============================================================

\section{Limitations}
\label{sec:limitations}

We acknowledge several limitations of our work:

\textbf{Dataset Construction.}
ONTO-Bench relies partially on curated questions from Wikipedia lists and 
authoritative sources. While we validate labels through cross-checking, 
the dataset may not capture the full distribution of epistemic challenges 
encountered in real-world applications. The ``unknown'' category is 
operationalized through curated open problems rather than a complete 
epistemological theory.

\textbf{Heuristic Detection.}
The ONTO epistemic oracle uses pattern-based heuristics for unknown detection 
rather than learned representations. While effective on our benchmark, this 
approach may not generalize to novel question formulations. We frame our 
contribution as demonstrating the value of explicit epistemic structure 
rather than proposing an optimal detection method.

\textbf{Baseline Configuration.}
We evaluate baseline LLMs without external tool augmentation (e.g., web search, 
retrieval) to ensure comparable settings with ONTO's core reasoning. 
Performance may differ with tool-augmented configurations.

\textbf{Dataset Scale.}
With 268 samples, ONTO-Bench is smaller than some established benchmarks. 
We prioritize label quality over quantity, but larger-scale evaluation 
would strengthen generalization claims.

\textbf{Evaluation Scope.}
We evaluate epistemic calibration through proxy classification tasks. 
Downstream impact on user trust, decision-making, and harm reduction 
requires separate investigation.

% ============================================================
% ETHICS STATEMENT
% ============================================================

\section{Ethics Statement}

\textbf{Intended Use.}
ONTO-Bench is designed for research evaluation of AI epistemic capabilities. 
It should not be used as the sole criterion for deployment decisions.

\textbf{Potential Misuse.}
High performance on ONTO-Bench does not guarantee safe deployment. 
Systems should undergo comprehensive evaluation before real-world use.

\textbf{Dataset Bias.}
Our dataset reflects Western scientific consensus and English-language sources. 
Epistemic norms may vary across cultures and languages.

% ============================================================
% REPRODUCIBILITY STATEMENT
% ============================================================

\section{Reproducibility Statement}

We provide complete materials for reproduction:

\begin{itemize}
    \item \textbf{Dataset}: Version-locked with SHA256 hash (see Appendix)
    \item \textbf{Code}: Full evaluation pipeline released
    \item \textbf{Baselines}: Model versions and prompts documented
    \item \textbf{Metrics}: Definitions and computation code provided
    \item \textbf{Random Seed}: Fixed at 42 for all experiments
\end{itemize}

Verification command: \texttt{python scripts/dataset\_version.py --verify}

% ============================================================

% ============================================================
% SANITIZED CLAIMS (use these exact phrasings)
% ============================================================

% MAIN CLAIM (Abstract/Intro)
\newcommand{\mainclaim}{%
ONTO significantly improves epistemic calibration on curated benchmark tasks, 
achieving substantially higher unknown detection recall compared to baseline LLMs.%
}

% CONTRIBUTION CLAIMS (Introduction)
\newcommand{\contribA}{%
We present ONTO, an ontological reasoning framework that augments LLM inference 
with explicit epistemic structure for improved uncertainty identification.%
}

\newcommand{\contribB}{%
We introduce ONTO-Bench, a benchmark for evaluating epistemic calibration 
comprising samples with explicit epistemic labels sourced from authoritative references.%
}

\newcommand{\contribC}{%
We demonstrate that explicit epistemic structure enables substantially higher 
unknown detection recall on our benchmark compared to baseline LLMs.%
}

% RESULTS CLAIM (avoid "dramatically", "revolutionary", etc.)
\newcommand{\resultsclaim}{%
ONTO achieves 96\% recall on unknown detection compared to less than 10\% 
for baseline LLMs on ONTO-Bench, suggesting that explicit epistemic structure 
can meaningfully improve uncertainty identification.%
}

% HEDGED CLAIMS (for discussion)
\newcommand{\hedgedclaim}{%
Our results suggest that ontological structure may provide a useful signal 
for epistemic calibration, though further work is needed to evaluate 
generalization to broader domains and real-world deployment scenarios.%
}

% ============================================================
% LIMITATIONS SECTION
% ============================================================

\section{Limitations}
\label{sec:limitations}

We acknowledge several limitations of our work:

\textbf{Dataset Construction.}
ONTO-Bench relies partially on curated questions from Wikipedia lists and 
authoritative sources. While we validate labels through cross-checking, 
the dataset may not capture the full distribution of epistemic challenges 
encountered in real-world applications. The ``unknown'' category is 
operationalized through curated open problems rather than a complete 
epistemological theory.

\textbf{Heuristic Detection.}
The ONTO epistemic oracle uses pattern-based heuristics for unknown detection 
rather than learned representations. While effective on our benchmark, this 
approach may not generalize to novel question formulations. We frame our 
contribution as demonstrating the value of explicit epistemic structure 
rather than proposing an optimal detection method.

\textbf{Baseline Configuration.}
We evaluate baseline LLMs without external tool augmentation (e.g., web search, 
retrieval) to ensure comparable settings with ONTO's core reasoning. 
Performance may differ with tool-augmented configurations.

\textbf{Dataset Scale.}
With 268 samples, ONTO-Bench is smaller than some established benchmarks. 
We prioritize label quality over quantity, but larger-scale evaluation 
would strengthen generalization claims.

\textbf{Evaluation Scope.}
We evaluate epistemic calibration through proxy classification tasks. 
Downstream impact on user trust, decision-making, and harm reduction 
requires separate investigation.

% ============================================================
% ETHICS STATEMENT
% ============================================================

\section{Ethics Statement}

\textbf{Intended Use.}
ONTO-Bench is designed for research evaluation of AI epistemic capabilities. 
It should not be used as the sole criterion for deployment decisions.

\textbf{Potential Misuse.}
High performance on ONTO-Bench does not guarantee safe deployment. 
Systems should undergo comprehensive evaluation before real-world use.

\textbf{Dataset Bias.}
Our dataset reflects Western scientific consensus and English-language sources. 
Epistemic norms may vary across cultures and languages.

% ============================================================
% REPRODUCIBILITY STATEMENT
% ============================================================

\section{Reproducibility Statement}

We provide complete materials for reproduction:

\begin{itemize}
    \item \textbf{Dataset}: Version-locked with SHA256 hash (see Appendix)
    \item \textbf{Code}: Full evaluation pipeline released
    \item \textbf{Baselines}: Model versions and prompts documented
    \item \textbf{Metrics}: Definitions and computation code provided
    \item \textbf{Random Seed}: Fixed at 42 for all experiments
\end{itemize}

Verification command: \texttt{python scripts/dataset\_version.py --verify}
